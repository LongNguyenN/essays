\documentclass{article}
\usepackage{graphicx} % Required for inserting images
\usepackage{geometry}
 \geometry{
 a4paper,
 total={170mm,257mm},
 left=20mm,
 top=20mm,
 }

\title{Jury Duty}
\author{Long Nguyen}
\date{March 2023}

\begin{document}

\maketitle

\section{Introduction}
Jury duty started at 9/26/22. There are some striking notes to take away from this experience. One is the amount of time taken to go through the entire procedure. The second was the trial itself. The last (for the sake of keeping a three paragraph structure) was the result of the trial. So let's jump into it with what the timing was like.

\section{Time}
Now, potential Jurors had to wake up incredibly early for this trial. We lined up at 8:15am to be shoved into a room, all awaiting selection. The selection held 70 of us, of which 14 would be picked, and the end jury would have 12 people. We were asked some questions, to which the defendant and prosecutor could evict us with consent of the judge. One person from the box was dismissed. The rest of us stayed and the the other 56 people dismissed. This was most likely an entire day out of people's time. The amount of wasted leisure and labor here is undeniable.

\section{Case}
The trial itself was a case of possible rape. It probably happened and women only lie in around 2-10 percent of cases [1]. This meant I should have just believed the women (if the source is to be believed and its hard to tell if its true). The case was Martin vs the state of Washington. It seemed the relationship started well enough until they fell out of it. She tried to leave him because of infidelity and he was then accused of raping her to assert dominance or pleasure or something. 

\section{Result}
The trial ended up as a hung trial. There was an even seven to seven split between the jurors. We could not convince each other of either side. After the trial we found some funny things like Karen not actually wanting to be there in the first place and the state calling her up as a witness; it would have been illegal for her not to attend the trial. Also the friendliness between the attorneys was a bit strange. We had Williams pointing at Wong and laughing a fake "haha" when we told them the vote was 7-5 in Martin's favor.

\section{Thoughts}
It was not certain if Wong believed Karen or was just representing her client, the state. There were some other mucky facts like Karen possibly not having citizenship and needing it though Martin. The whole atmosphere was outside of expectations too. The jurors would joke around a lot during deliberation; to the point of wondering if it would bother the procession outside. The attorney, William, also got into weird verbal fights with jurors and one of the detectives as well. At one point, they were arguing about whether several meant 2 or 3.

\section{Conclusion}
Let me try to end this in 4 sentences (counting this one). The experience was not what I expected and I would not like to do another one. It was time consuming and ultimately a waste of time. We decided nothing and even if we did, whatever punishment dulled out by the judicial system would not have been appropriate; it would not have helped anyone. I am glad I was able to go through the entire trial, just to get an understanding of what our judicial system is like. I think that I now understand that it is just wack.

\section{Sources}
1. https://web.archive.org/web/20180101025446/https://icdv.idaho.gov/conference/handouts/False-Allegations.pdf
2. A lot of this is just from my memory and journal entries

\end{document}

