\documentclass{article}
\usepackage{graphicx} % Required for inserting images
\usepackage{geometry}
 \geometry{
 a4paper,
 total={170mm,257mm},
 left=20mm,
 top=20mm,
 }

\title{Summary of Des Capital}
\author{Long Nguyen}
\date{August 2023}

\begin{document}

\maketitle

\section{Chapter 15}
A whole series of bourgeois political economists insist that all machinery displaces workmen and simultaneously sets free an amount of capital adequate to employ the same identical workmen. The machine frees the workmen of their means of subsistence and converts that into the capital of the owner. "If you abolish the knife, you throw us back into the depths of barbarism." Marx finds that in the age of the machine, the number of laborers end up amounting fewer than the number of slaves.

"In the hardware manufactures of Birmingham and the neighbourhood, there are employed, mostly in very heavy work, 30,000 children and young persons, besides 10,000 women. There they are to be seen in the unwholesome brass foundries, button factories, enamelling, galvanising, and lackering works." "Work from 5 in the morning till 7 in the evening is considered 'reduced' and 'moderate'. Both boys and girls of 6 and even of 4 years of age are employed." (pg 321) Marx here describes terrible injustices. He says girls learn fit into a context of men before they are even aware they are women. This just shows how working people from a young age can have dehumanizing effects. He even describes cases of language being affected as the young people do not get an education and learn from adults who grew up in the same situations they did. 

The labour required for machinery stretches workers. This is so that it requires women and children worker. Work hours are also stretched into overtime and night hours. This reaches a natural breaking point. The industrial revolution saw the production of ever cheaper machines that began using steam power. The older machines were sold for cheap to every smaller amounts of capitalists. The acumulation of capital into ever lesser hands intensified. The railroads and telegraph make purchasing orders sporadically a possibility. So there are busy seasons in which overtime is natural and employment becomes irregular.


The factory acts in England included loose and easily broken provisions to protect workers. These were sanitary clauses that were limited to white walls, some other cleanliness measures, ventiliation, and protection against dangerous machinery. "27,800 artisans, hitherto breathing through protracted days and often nights of labour, a mephetic atmosphere, and which rendered an otherwise comparatively innocuous occupation, pregnant with disease and death." English doctors recommend at least 500 square feet per worker for good conditions. This would right away put many small capitalist out of the running. This would only the big capitalists to remain. The factory act also included education clauses. It was found that many of these clauses were not followed strictly. Children would receive half days of education only. But strangly it was found that these children preformed better in schooling. It is speculated that the half day gives them a rest from work, so they are eager to learn. "They become recruits of crime. Several attempts to procure them employment elsewhere, were rendered of no avail by their ignorance and brutality, and by their mental and bodily degradation. "Parents must not posses the absolute power of making their children and young persons machines to earn so much weekly wage." In actuality it is capitalism that creates the economic circumstances that have parents doing this. Of course the family unit should not be held religiously either. Business owners argue that the factory act harms them and restricts their free trade. They argue that the acts would have the labor of boys be put elsewhere other than in their factories. This they argue is especially unfair to the larger business which it only applies to as it hurts them in competition. They also argue, aggregiously, that it would be least favorable to all "health, comfort, education, and general improvement of the people." This is an argument made in general and in specific by Messrs Cooksley of Bristol, a nail and chain manufacturer.

\section{chapter 16}
Productive labor is surplus labor that benefits capitalists. The capitalist mode of production requires an increasing of surplus value over necessary labor power. A general attitude shift to the capitalist mode of production would mean maximizing surplus value and thus extending the working day. Once the working day could no longer be extended, work would then become more productive or intense to increase surplus value. 

Capital requires the domination of nature. It is not enough for their to be good soils Marx says. He points to Industry requiring this dominance as well and the first examples in history he brings up is irrigation in Egypt, Lombardy, Holland, India, Persia. Marx uses an example of the people of the Asiatic Archipelago who use sago that grows wild in the forest. One sago tree can yield 300 pounds, so "people go into forests, and cut bread for themselves, just as with us they cut fire-wood." But in the east, a working person may require 12 working hours to satisfy all their wants. Nature could provide for this worker plenty of leisure time, but this time is spent instead creating surplus value to be given to a stranger. 

Marx sites Ricardo and calls Ricardo a mercantile. Ricardo's theory of profit is that it is inherent in the capitalist mode of production. He does think that it comes from labor, but does not touch too much on it. He shys away from it, afraid of the implications it could entail. Marx also mentions John Stuart Mill, another Mercantile. Mill goes to say that profit does not come from the laborer, but from the fact that the product of labor lasts longer than the time required to make it. This is obviously wrong. 

\section{chapter 17}
A more intense working day means more labor power is put in and thus more products are produced with the length of the day remaining constant. 

The working day produces value in proportion to its length. Discussion of a change in the working day assumes that productiveness and intensity of labor is constant, but it often is not as we have seen with the example of workers increasing productivity with a shortened working day.

The minimum length of a working day is fixed by the laborer's need to buy the neccesities. Thus a shortening of the working day to that amount would mean eliminating surplus value, which would not be allowed under regime of capital. In a capitalist society, the spare time of one class is aquired from the convertion of another classes life-time into labor-time. 

\section{chapter 18}
The rate of surplus-value is as follows
surplus-value/variable capital = surplus value/value of labor power = surplus product/total product.

A derivative formula that obscures the relationship in capital's favoring  by looking like the capitalist and the laborer share in proportion their equal contribution.
surplus-labor/working-day = surplus-value/value of the product = surplus-product/total product.

\section{chapter 19}
The value of labor is the natural price and the market price is the fluctuation above or below natural price. Labor is the measure of value, but in itself has no value. In classic economics, demand and supply determine the price of a good. But then what determines the point of equilibrium? This ends up being the value of labor or the natural price. The money relation hides the full value that the laborer provides for the boss. Because they are paid a wage, they do not know how much of their wages goes just into their reproduction and how much goes into profits. It is then incredibly important for capitalists to have money so that they can pay wages. So labor must be transformed into wages for the system to work. This contrasts to trading regular commodities because the value of all commodities in a regular trade is in full display. 

\section{chapter 20}
With hourly labor, capitalists can play with the hours. They can lengthen or shorten it at will. They can play this in their favor as they see fit. Overtime pay is also taken into consideration when calculating the value of the working day. The lower the price of labor, the greater the quantity of labor, or the greater the number of labor hours are required to make up for the low price of labor. The increase in working time actually decreases the price of labor and decreases the day's wage. 

\section{chapter 21}
There is another form of wages which is piece wages. The difference in forms of the wages by no means changes their essential nature. With piece wage, the hourly time necessary to create a certain piece can be experimentally determined. If a worker is found not to be able to meet the average requirement of time to create a piece, then they are dismissed. In this way, the capitalist can even better exploit the worker because the capitalist knows the worker is indeed making the most use of the time being paid. The laborer here is then in their own interest would increase the intensity of their labor and their own working day. 

\section{chapter 22}
The sum of money varies according to differing national values. Nations with more developed modes of capitalist production will have a lesser ralative value. The nominal wages will be higher in first nations than in second nations. The price of labor compared with surplus value and the value of the product is higher in the second nation. 

\section{chapter 23}
Simple reproduction. Society cannot cease to produce just like it cannot cease to consume. The condition of reproduction is also the condition of production. Simple reproduction is the repetiion of production. The worker is alienated from his work at the sale of his labour power. The capitalist receives from paying for labour power both the value of labor and the reproduction of labor from the payment for the reproduction of labor. What the laborer spends on anything outside of reproduction is unproductive consumption to the capitalist. The laborer is bought into capital even before and inbetween selling his labor power.

\section{chapter 24}
The conversion of surplus-value into capital. The capitalist uses the labor of one year to create the capital to pay the labor of the next year. The neccesity to seperate property from labor. Capital's justification for hoarding wealth is that this saves the wealth from being consumed so that it can be put into purchasing more labor. The capitalist is required to expend capital to make more capital because of the vicious competition. Accumulation for accumulation's sake. The cost of labor cannot be zero but the capitalist wishes it to approach zero as if it being some mathematical limit. capitalism is a social wealth. There is an idea of a lbaor fund that takes the earnings from all labor and pools them. From this labor fund it is possible to find the average wage of laborers by dividing the labor fund by the ammount of laborers.

\section{chapter 25}
Capital can be divided into constant and variable which is capital for raw resources and capital for labor power respectively. Or in another word means of production and labor power. There is value composition or organic composition and techincal composition of capital. Capital requires laborers. The increase in demand for labor means higher wages and a further increase in reproductive resources going to labor and thus more laborers. This means that increase capital results in increase labor/proletariate. A rich man with no laborers is a laborer and so a rich man with many laborers is a richer man. The laborer makes a rich man so the more laborers, the more rich men there will be. In any rich country, the poor should never be idle and should spend any money they receive. This is all said from men like John Bellers and Bernard de Mandeville. The rich are those who do not labor and those with property then can gain significant inffluence over those without. Accumulation can go about in two ways: either raises in wages do not interfer with accumulation or they do. In the case they don't, everyone wins, but eventually they do. When accumulation cannot increase, then labor's wage is decreases. This means that wages are dependent on accumulation. A lack of accumulation is in fact due to a shortage of labor and not because of high wages. The price of labor is kept in check to assure the accumulation of capital. The rise in wages are allowed so long as they prolong and propogate the system. As productivity increases, the value of the product diminishes and so does the proportion of capital that goes into variable capital compared to constant capital. A decrease in the proportion going into variable capital does not mean a decrease in the absolute value of variable capital. 
Centralization as distinct from accumulation and concentration. The attraction of capital by capital, Marx says that he will not develop it here. Cheepness of commodities depends on the productivity of labor. The smaller capital will lose out to larger capital. The minimum capital needed to start business also increases. This all leads to further centralization. 
Every historic mode of production has its own rules for population. A surplus population is necessary for the accumulation of capital. The constant accumulation could lead to technical advances that require more labor for production, this requires a mass of reserve labor ready to be plugged in. It is necessary for a reserve so that labor is not taken from other sections of production. Effect becomes cause and there is a periodicity called an industrial cycle. If immigrants are kicked out, production would suffer, this would be a plausible excuse to raise prices. After prices are raised, labor can be kicked back up again without prices going down. Machinery moves variable capital to fixed capital and does not set workers free. So the absolute capital increases but the demand for labor does not. As soon as labor realizes that supply and demand of labor only increases the intensity of labor once supply is down, do laborers unite the employed and undemployed, but as soon as that happens, the state is employed to quell it.i

Surplus population can come in the form of part time workers, underemployed workers, or during times of crisis and there is a change in industry. For this reason there can be cries for the need of more hands while there being unemployed workers at the same time. They are all specialized in a particular industry. Pauperism is another key indicator of capitalism. These are the beggers, the orphans, the widows which are there ready to be put to labor and thrown away otherwise. A quote from a bourgeois Destutt de Tracy "In poor nations the people are comfortable, in rich nations they are generally poor."

\section{chapter 27}

\section{chapter 28}
In the 16th century Henry VII put in bloody legislation that would put the working laborer back into their place. This legislation stipulated that an old beggar would receive a beggar card while a healthy vagabond caught begging would be whipped until he went back to labor. Peoples who were forced off their lands and had them expropriated were then whipped into submission by grotesque laws until they would accept wage labor. Elizabeth 1572, beggars found above the age of 14 were to be marked on the ear and put to work. Those above 18 were to be executed. 

A maximum wage will be dictated by the sate but no minimum before 19th century. In 1796 Whitebread proposed a legal minimum. Pitt apposed this but admited the condition of the working poor was cruel. A decree of 1791 declared strikes to be illegal and put the fight over wages into the comfortable purview of the capitalist. More over it limited the workers abilities to act in common and for their own self intrests. 

\section{chapter 29}

\end{document}

