\documentclass{article}
\usepackage{graphicx} % Required for inserting images
\usepackage{geometry}
 \geometry{
 a4paper,
 total={170mm,257mm},
 left=20mm,
 top=20mm,
 }

\title{}
\author{Long Nguyen}
\date{August 2023}

\begin{document}

\maketitle

\section{Intro}
Many ecologists, activists, and academics argue that economic growth is the cause of our environmental crisis. One solution proposed is degrowth of the economy. Too often, this argument leads away from class struggle and focuses instead on destructive consumption patterns. This blames individual consumption patterns say using platic instead of paper straws. We socialists should take a clear stance that the climate crisis is a result of capitalist's insatiatble pursuit of profit and that the only solution is a socialist one in which human needs, including a sustainable relationship with nature, should take priority over private greed.

\section{Boom and Bust}
The pursuit of profit under capitalism is relentless and requires continuous growth [1]. Here I'm going to take a small detour into an analysis of capital. What differentiates capital from money is that capital's main function is to be used as an investment to increase itself. In capitalism, the entire economy depends on the investment of capital. Marx put this in a his suscinct equation M-C-M' where money, itself a commodity, is traded for some non money commodity and this commodity is then traded to make more money. So this is where growth comes from. The desire of the entire economy to produce more money than was originally invested. It follows that a growth in money requires a growth in goods and services to exchange for that money. The growth of goods would come from a growth in extraction of natural resources. To reiterate, capitalism fundamentally requires growth.

This basic analysis is not the only reason capitalism requires growth. Further down the line, as competition creates pressures for firms to increase productivity and lower costs, workers would find themselves losing employment as the firms require fewer workers due to increased productivity. The economy would then need to grow inorder to create more jobs to support these unemployed workers.

A little more theory. Rosa Luxemburg asked the question, in Marx's M-C-M' equation, if applied on an economy wide scale, where does that extra money come from? Her answer is from credit. The economy grows from credit created out of thin air when banks approve of a loan. Now the banks expect their loans to be paid back with interest. This then requires the ecnomy to continue to grow in order to pay back said loans with interest. A lost in faith in debtors and in the ability for loans to be paid back, you can imagine, would put a halt to this entire system. As an aside, a system in which profits to initial investors are predicated on recruitment of future investors is in fact called a ponzi scheme. So credit and interest is yet another reason that capitalism requires growth. 

When capitalism does not get the growth it requires or even if the growth is not fast enough, the system crashes. This is the so called "business cycle". It is a cycle of booms and busts that leaves workers ever more desperate and capitalists richer. This is not a stable or sustainable system.

\section{Degrowth in an unequal world}
Economic growth has long been treated as the core policy strategy to increase living standards [3]. The development and  of countries in the global south such as Brazil, India, and China has seen the living conditions of many people change dramatically. Though the key to this economic growth has been fossil fuels, which increases the amount of carbon emissions.

The global north is responsible for  causing climate change which the South is likely to suffer from with many terming this 'carbon colonialism.' Although the tables have flipped as in 2017 the global south emmited 21 Gt C02 while the north emitted 13.7 Gt. So the south in catching up with development with the north has surpassed it in c02 emissions. 

Historically low income countries have contributed very little to climate change, but will be the ones most severly affected. Adverse macroeconomic effects will be felt by countries with hot climates which accounts for most low income countries. A rise in temperature lowers per capita output in countries with high average temperatures. And in areas with hot climates, the high temperatures will reduce agricultural output, lower productivity of workers, and damage health. The same can be said for island nations which have contributed very little to climate change but will most severly be affected by rising sea levels.

Capitalists also seek to externalize costs [2]. These are consumption, production, and investment decisions that affect people not directly involved in the transaction. An example is east palestine. Norfolk Southern chose to externalize costs when they cut corners on safety and ignored rail workers who wanted increased standards of living as well as better working conditions. These workers were asking for paid sick days, pay increases, more paid time off, better working ocnditions, and more flexible schedules. These workers received non of this. Their strike even warned of safety hazards and train derailments which happened in east palestine. The cost of the train derailment will fall on the shoulder of the people who suffer from the incident. Also much profit will go directly to a wealthy few ceos at the top of norfolk, rather than the workers.

The constant need for more profits means accumulating resources in continously more destructive ways. This leads to the depletion of soils, minerals, forests, and the life in our oceans. The destruction is a contradiction in which this system undermines its own wealth. Further, capitalists are running into ecological barriers as seen with environmental disasters like the shutdown of the power system in texas or the global pandemic.

Capitalists speculate through the market which has little relation to actual value; Marx calls this fictitious capital. This shows how the capitalist conception of value is further unhinged from reality as they destroy all real sources of wealth in an attempt to squeeze out every last ounce of profit they can.

Another contradiction capitalists run into is that they are driving down the share of wealth that goes to the workers. The workers then are not able to buy all that the capitalists produce. This growth leads the economy into a crisis and recession.

If growth were instead put into productive investments it could increase the living standard of the working class. An example is after WW2 when wealth was put to pensions, public health and education, and welfare protections. Of course this was done as a capitalist defense mechanism against working class revolution.

Today we see growth coming from the shrinking of wealth going to the working class as well as the increasing consumption powered by debt. This kind of investment leads to little increases in living standards for workers.

The recovery from the 2008 recession was largely joyless as there was impressive growth in the economy while pay remained low and housing was in distress. Wages only really recovedered to 2008 levels at the end of 2019 only to be met with another crisis.

The 2008 recession also saw the rise of the gig economy [4]. These are sharing apps such as Airbnb and Uber. As people found themselves out of work or in need of additional cash, they toke to these small "gigs" for suplemental income.

Yet as people suffer, the banks received massive bailouts. The biggest part of the Troubled Asset Relief Program or TARP was to rescue banks which amounted to \$ 236 billions going to over 700 banks. And many of these banks have yet to pay back their debt to the US government. One example is OneUnited which hasn't made dividend payments since 2009. It owes \$8.7 million in unpaid dividends and \$12 million in principle [5]. And I want to remind everyone that these recessions are a regular happenstance that has even been termed the business cycle.

Even with all of the problems of growth, degrowth is not the answer. And it certainly is not a movement that would move the working class into solidarity and action.

A recent UN report shows that globally the top 1 percent of earners are responsible for a yearly per capita average of 74 tons of c02 per year. Meanwhile, the bottom 50 percent of earners put out 0.7 tons.

According to one study, examining 150 countries over the period of 1960-2008, a 1 percent increase in GDP meant on average a 0.73 increase in carbon emissions, while a 1 percent decline in GDP meant only a 0.4 percent decrease in carbon emissions. Less consumption by itself cannot reduce carbon emissions. So without a planned transition to a sustainable way of life, there is no way of decreasing emissions. Then the growth degrowth debate is worthless unless it is linked to brining an end to capitalism.

The purpose of the economy under socialism would be to fulfill human needs in a sustainable way. This means public ownership of key sectors of the economy such as the energy industry, transporation, agribusiness, and production overall. Socialist want a better life for the vast majority on this earth. Many, even in rich countries, are in poverty or barely able to keep their heads above the water; they do not have access to decent housing or healthcare or have no economic security. This is unjustifiable in a world of incredible abundance. For these reasons, we reject attacks on working-class living standards, even those that are introduced with an environmental veneer, e.g. water charges or carbon taxes.

\section{Waste}
Capitalist production involves enourmous waste.
\begin{itemize}
	\item 690 million people around the world went hungry in 2019. Yet during the pandemic the closure of restaurants and other disruptions caused the widspread dumping of perfectly good produce. Even in normal times, enough food is produced to feed everyone, but one third of this food is lost or wasted. A large part of this is because of the commodification of food. Agribusniness leaves foot to rot in the fields to keep prices high, supermarkets throw out good food that they don't think will sell, good food is discarded because its size or shape make it unmarketable.
	\item In 2020, approximately \$ 569 billion was spent on advertising. The vast amount of this money is wasted as it is spent not to inform us, but to convince us to buy as much as possible or to choose between one identical brand of a produce over another. Advertising often preys on our axieties and insecuities in order to create false needs that can be "solved" through consumption.
	\item Capitalism does not produce for need, but for profit. The packaging industry is now the third largest on earth and much packaging is not mainly functional but a form of product promotion. Packaging costs amount to somewhere between 10 percent and 40 percent of the total product cost.
	\item Planned obsolescence means that products are conciously not built to be durable and must be frequently replaced. This includes fast faction and electronics with batteries. 
	\item Then there are the products that are of no use to working people. These are armaments, luxury goods like private jets, crypto currencies such as bitcoin consume more energy than all of argentina, that's a country of 45 million people.
	\item competition between firms means that research and development efforts are often duplicated
\end{itemize}

\section{A sustainable future would need socialist planning}
Some argue that a simple transition to renewable energy will solve the ecological problems we face. But this wont happen under capitalism. Even if this were achieved, we would still face looming ecological catastrophes as capitalism has already exceeded a number of planetary boundaries for safegaurding a safe environment for human civilization on earth.

These planetary boundaries are species extinction, soil degredation, and deforestation to name a few. Their source in the increasing scale and intensity of human incursion in to nature. This undermines the basis for our own existence on this planet.

In summer camp comrad Evee brought up excellent points in her leadoff, among them was that 95 percent of the US population has been affected by climate change. She also mentions that 20-30 percent deforestation of the Amazon rain forest is another one of these tipping points. It would result in more forest fires which would release carbon into the atmosphere.

Technological changes alone will not be the solution. Under capitalism, more efficient energy use due to techonology will only result in further expansion, so paradoxically technological development often results in a net increase in the amount of energy used.

Comrad Liv from Vashon Island pointed out that degrowth can be co-opted by capitalists in the same way regenerative agriculture has been co-opted. It either offers no solutions or it a turn to individualistic solutions to system problems.

Comrad Rose added to this and is able to make the connection between degrowth and intersectionality. These movements may be helpful for some people and maybe even a solution to some problems, but are also a distraction from class struggle.

You can't have infinite growth on a finite planet.

Even so, we do not advocate for degrowth. There are a few reasons for this. One is that it is not a reasonable thing to ask of regular people. We do not wish for them to think that their standards of living will decrease. We do not wish for their standards of living to decrease. Second, we do not need degrowth. It can be more nuanced than that. Certainly destructive sectors of the economy like fossil fuel companies could use some degrowth, but then the green energy sectors would need to grow. We could also use the resources at our disposle in a smarter way. We could limit capitalist waste and expand public transportation for example. Then there's also the fact that such solutions would not even be allowed under capitalism.

Socialists stand for massive investments in low carbon jobs and sustainable infrustructure as well as the introduction of a four-day work-week with no loss of pay. This would solve unemployment as it would require redistributing work but also give workers the time to participate in politics and economic decisions. This is what a climate solution would look like under socialism. And we need a socialst word to make it happen.

This will still pose complex questions about how products, industries, and practices can be maintained. But these are best resolved on the basis of democratic discussion in a society founded on equality and solidarity.

One discussion question I want to raise is what is our Tolerance level for degrowthers in SA?

\section{Sum up}
One more contribution I want to mention is that from Comrad Kshama. She emphasizes that it is no coincidence that intersectionality and degrowth have popped up when they have. They are attempts to divert attention from class struggle. She asks "can these ideas even be implemented in a useful way under capitalism"? This is retorical as she her answer is no. She continues, degrowth does not go into politics, they run because it is a move away from politics and it is easier. This is comparable to the BLM struggle. That movement saw young people radicalizing around racism, but lost because of a lack of class struggle and the right wing was able to move into the vacuum resulting from that lost. She concludes that degrowth is a false move and a wasted effort because under capitalism it would not happen anyways.

\section{sources}
\begin{verbatim}
1] https://mronline.org/wp-content/uploads/2019/06/why_capitalist_economies_need_to_grow_-_for_green_house_-_10_10_14.pdf

2] https://www.imf.org/en/Publications/fandd/issues/Series/Back-to-Basics/Externalities

3] https://www.tandfonline.com/doi/full/10.1080/01436597.2021.1954901

4] https://www.statista.com/study/67456/the-rise-of-the-gig-economy-in-the-us/

5] https://www.propublica.org/article/the-bailout-was-11-years-ago-were-still-tracking-every-penny
\end{verbatim}

\end{document}

